%\chapter{绪论}
\chaptera{绪论}
\section{研究工作的背景与意义}


可再生能源的开发利用既缓解了能源危机,降低了污染物的排放\citing{王永华2013现代电气控制及, 刘海龙2010浅谈电气自动化的现状与发展方向, 王磊2011电气自动化控制设备可靠性探究},也提高了电力系统的经济性。然而,由于风光发电系统本质上对气象条件的依赖,其间歇性和波动性会对电力系统的功率平衡、\citing{张燕2013电气自动化在电气工程中的应用探讨,陈众励2007建筑电气节能技术综述}网络损耗等产生诸多负面影响。近年来,不断涌现的电动汽车为这一现状带来了转机\footnote{脚注序号“\ding{172},……,\ding{180}”的字体是“正文”,不是“上标”,序号与脚注内容文字之间空1个半角字符,脚注的段落格式为:单倍行距,段前空0磅,段后空0磅,悬挂缩进1.5字符;中文用宋体,字号为小五号,英文和数字用Times New Roman字体,字号为9磅;中英文混排时,所有标点符号(例如逗号“,”、括号“()”等)一律使用中文输入状态下的标点符号,但小数点采用英文状态下的样式“.”。}。电动汽车具有分布式储能单元的特性\citing{王永华2013现代电气控制及},对其进行合理的充放电调度不仅能够有效控制其无序充电所带来的负面影响,还能丰富电力系统运行和控制手段,如:削峰填谷,提高设备利用率;跟踪可再生能源,提高其接入量\citing{王磊2011电气自动化控制设备可靠性探究};为系统提供调频等辅助服务,提高系统的运行可靠性。

\section{电动汽车分群调度策略的国内外研究历史与现状}
电动汽车充放电调度方面的相关研究已日渐成熟,现有的充放电调度模型并不适合在实际调度中应用。目前对电动汽车的充放电
调度从研究层面可分为2类。

\section{本文的主要贡献与创新}
本文提出了一种电动汽车分群调度策略,在日前策略制定阶段,基于描述电动汽车特性的4个判别量对其进行分群

\section{本论文的结构安排}
本文的章节结构安排如下:
