
\chapter{算例分析}
\section{第1节}
本文以修改的IEEE-33节点配电网为例验证所提分群调度方法的有效性和采用区间优化的必要性
\subsection{第2节}
本文取每辆电动汽车的每\SI{100}{km}电能损耗为$1$,充放电功率为$2$;波动系数取$0.15$;负荷功率的标准差取期望值的$2\%$;概率水平取$95\%$;不

数值单位应使用\textbf{siunitx}宏包,具体用法可使用命令 texdoc 查看。简单介绍如下:
\begin{quote}
$\backslash$SI\{\#1\}\{\#2\},其中\#1为数值,\#2为单位。

共有$\backslash$SI\{100\}\{kg\} $\rightarrow$ 共有\SI{100}{kg}。
\end{quote}

\subsection{数值算例与分析}
算法如下:

\begin{algorithm}[H]
 \KwData{this text}
 \KwResult{how to write algorithm with \LaTeX2e }
 initialization\;
 \While{not at end of this document}{
  read current\;
  \eIf{understand}{
   go to next section\;
   current section becomes this one\;
   }{
   go back to the beginning of current section\;
  }
 }
 \caption{How to wirte an algorithm.}
\end{algorithm}


\section{方程的求解}

\section{本章小结}
。